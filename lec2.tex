\chapter{Klassifizierung partieller Differenziagleichungen}
\lecture{2}{26.04.2021}{DGLs}

Im Rahmen der Strömungsmechanik sind verschiedene gekoppelte (partielle) Differentialgleichungen relevant.
Insbesondere die Kontinuitäts- Impuls- und Energiegleichung müssen fast immer im Zusammenhang betrachtet werden.
Außerdem sind diese Gleichungen teilweise von verschiedenen Graden und im Allgemeinen hochgradig nichtlinear.

\section{Transportprozesse in Fluiden}
Wir betrachten eine beliebige physikalische Größe $u$ und deren Transport in einem Fluid.

Man nennt die folgende Gleichung Konvektionsgleichung:
\[
\frac{\partial u}{\partial t} + \vec{v} \cdot \nabla u = 0
.\]
Dabei ist $\vec{v}$ der Geschwindigkeitsvektor der Strömung.
Man nennt solche Gleichungen \textbf{hyperbolische} Gleichungen.
Ein beispiel ist eine einfache Welle, bei der die Größe mit dem Geschwindigkeitsfeld des Fluids einfach mittransportiert wird.
Die Lösung einer solchen einfachen Wellengleichung ist $u = u_0(\vec{x} - \vec{v}t)$.
\newline

Betrachten wir nun die sog. Diffusionsprozesse.
Auch ohne Konvektion tritt oft eine zeitliche Abhängigkeit in der Lösung auf.
Aufgrund von Konzentrationsgradienten einer Größe kommt es zu ihrem Transport durch Diffusion.
\[
	\frac{\partial u}{\partial t} = \nabla \cdot (\nu \nabla u) , \nu > 0
.\]
Für $\nu = \text{const.}$ ergibt sich 
\[
\frac{\partial u}{\partial t} = \nu \nabla^2 u
.\]
Dabei ist $\nu$ %TODO 
Die Diffusionsgleichungen sind \textbf{parabolische} Gleichungen.
Die Wirkung von Diffusionsprozessen hängt stark von der Entfernung zur Ursache ab.
Im Moment nach der Anfangsbedingung bewirkt ein Gradient in der Größe $u$ bereits eine Änderung in allen Punkten im Berechnungsgebiet
(wenn auch mit abnehmender Entfernung nur unwesentlich größer als 0).
Die Lösungen einer solchen parabolischen Gleichung ist also keine Welle.
Die Ausbreitung der betrachteten Größe erfolgt aufgrund von Gradienten und in alle Richtungen!
\newline

Mit dem Diffusionsgleichgewicht für $t \to \infty$:
\[
	\nabla \cdot (\nu \nabla u) = f(\vec{x})
.\]
lassen sich weitere Effekte wie Reibung in Fluiden beschreiben.
Man nennt solche Gleichungen \textbf{elliptische} Gleichungen.
Da diese Gleichungen keine Zeitabhängigkeiten mehr besitzt, kann von einer Wellenlösung keine Rede sein.

\section{Gleichungen zweiter Ordnung}
Differentialgleichungen 2. Ordnung haben die allgemeine Form
\[
	a \frac{\partial ^2u}{\partial x^2} + 2b \frac{\partial^2 u}{\partial x \partial y} + c \frac{\partial^2 u}{\partial y^2} = f(x,y,u,\partial_x u, \partial_y u)
.\]
Man kann nun Störungen dieser Gleichung betrachten und erhält so eine Fluktuation $u'$ und eine linearisierte Störungsgleichung:
 \[
	 a_0 \frac{\partial^2 u'}{\partial x^2} + 2b_0 \frac{\partial^2 u}{\partial x \partial y} + c_0 \frac{\partial^2 u}{\partial y^2} = 0
.\]
Dabei sind jetzt $a,b,c$ konstant.
Man diese Linearisierung selbst für die allgemeine Form stets durchführen!

Für diese gestörten Gleichungen kann man dann eine Wellenlösung suchen:
\section{Systeme erster Ordnung}

\section{Rand- und Anfangsbedingungen}

