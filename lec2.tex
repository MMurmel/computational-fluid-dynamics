\chapter{Klassifizierung partieller Differenziagleichungen}
\lecture{2}{26.04.2021}{DGLs}

Im Rahmen der Strömungsmechanik sind verschiedene gekoppelte (partielle) Differentialgleichungen relevant.
Insbesondere die Kontinuitäts-, Impuls- und Energiegleichung müssen fast immer im Zusammenhang betrachtet werden.
Außerdem sind diese Gleichungen teilweise von verschiedenen Graden und im Allgemeinen hochgradig nichtlinear.

\section{Transportprozesse in Fluiden}
\subsection{Konvektion}
Wir betrachten eine beliebige physikalische Größe $u$ und deren Transport in einem Fluid.

Man nennt die folgende Gleichung Konvektionsgleichung:
\[
\frac{\partial u}{\partial t} + \vec{v} \cdot \nabla u = 0
.\]
Der Term $\frac{\partial u}{\partial t}$ beschreibt die lokale zeitliche Ableitung.
Den Term $\vec{v} \cdot \nabla u$ nennt man „konvektive Ableitung“,
wobei $\vec{v}$ den Geschwindigkeitsvektor der Strömung darstellt.

Konvektionsgleichungen bilden den Grundtyp der \textbf{hyperbolischen} Differentialgleichungen.
Diese erstrecken sich darüber hinaus auch auf die Erhaltungsgleichungen,
die eine allgemeinere Form der Konvektionsgleichungen bilden und für die Strömungsmechanik äußerst bedeutend sind.
Sie besitzen die Form:
\[
	\frac{\partial u}{\partial t} + \frac{\partial f(u)}{\partial x} = \frac{\partial u}{\partial t} + \underbrace{f'(u)}_{\mathrel{\widehat{=}}v} \frac{\partial u}{\partial x} = 0
.\]
Darauf soll an dieser Stelle jedoch nicht weiter eingegangen werden.
\newpage
Konvektionsgleichungen besitzen sogenannte „einfache Wellenlösungen“ der Form:
\[
	u = u_0 \left( \vec{x} - \vec{v}t \right)
.\]
Die Lösung für $u$ ist also eine Lösung für die Rand- und Anfangswertbedingungen $u_0$,
die mit der Zeit einfach durch die Geschwindigkeit $\vec{v}$ im Raum verschoben wird.

\begin{merke}
	\begin{enumerate}
		\item Hyperbolische Gleichungen besitzen eine zeitartige Variable (in den obenstehenden Gleichungen tatsächlich immer die Zeit).
		\item Die Lösungen für hyperbolische Gleichungen können komplett aus Wellenlösungen zusammengesetzt werden.
	\end{enumerate}
\end{merke}

\subsection{Diffusion}
Betrachten wir nun die sog. Diffusionsprozesse.
Auch ohne Konvektion tritt oft eine zeitliche Abhängigkeit in der Lösung auf.
Aufgrund von (Konzentrations-) Gradienten einer Größe kommt es zu ihrem Transport durch Diffusion.
Die folgende Gleichung nennt man Diffusionsgleichung:
\[
	\frac{\partial u}{\partial t} = \nabla \cdot (\nu \nabla u) , \nu > 0
.\]
Man nennt $\nabla \cdot (\nu \nabla u)$ den Diffusionsterm und $\nu$ den Diffusionskoeffizienten.
$\nu > 0$ folgt aus dem zweiten Hauptsatz der Thermodynamik, wobei  $\nu = 0$ für den idealisierten Fall möglich ist.
Für $\nu = \text{const.}$ ergibt sich 
\[
\frac{\partial u}{\partial t} = \nu \nabla^2 u
.\]
Im Gegensatz zu Konvektionsgleichungen fällt auf, dass es keine ausgezeichnete Transportrichtung oder -geschwindigkeit gibt.
Außerdem gibt es eine Momentanwirkung in \textit{jedem} Bereich des Rechengebietes.
Von einer Wellenlösung kann daher keine Rede sein.
\newline
Diffusionsgleichungen bilden den Grundtyp der \textbf{parabolischen} Differentialgleichungen.

\begin{merke}
	\begin{enumerate}
		\item Parabolische Gleichungen besitzen eine zeitartige Variable.
		\item Es existieren keine Wellenlösungen (nur sog. „degenerierte Welle“, die hier aber nicht weiter relevant sein sollen.)
	\end{enumerate}
\end{merke}

\subsection{Elliptische Gleichungen}
Betrachten wir nun einen Diffusionsprozess, der sich asymptotisch an ein Gleichgewicht annähert, also für den für $t \to \infty$ gilt:
\[
	\nabla \cdot (\nu \nabla u) = f(\vec{x})
.\]
Dabei ist $f(\vec{x})$ ein sogenannter „Quellterm“, der beschreiben kann, dass keine Gleichverteilung vorliegen muss (sondern nur \textit{zeitliches} Gleichgewicht).
Sie bilden den Grundtyp der \textbf{elliptischen} Differentialgleichungen.
\begin{merke}
	\begin{enumerate}
		\item Elliptische Gleichungen besizten \textit{keine} zeitartige Variable.
		\item Elliptische Gleichungen besizten \textit{keine} Wellenlösungen.
	\end{enumerate}
\end{merke}

Bisher haben wir die Typen von DGLs anhand ihrer Eigenschaften bezüglich Existenz von zeitartigen Variablen und Existenz von Wellenlösungen unterschieden.
Diese Unterscheidung fiel an dieser Stelle leich, da wir die „Reinkulturen“ hyperbolische, parabolische und elliptischer Gleichungen betrachtet haben.
Auch wenn diese Einteilung im Allgemeinen weder so einfach noch eindeutig ist, wollen wir zunächst weiter versuchen ihr zu folgen.

\section{Gleichungen zweiter Ordnung}
Differentialgleichungen 2. Ordnung haben die allgemeine Form
\[
	a \frac{\partial ^2u}{\partial x^2} + 2b \frac{\partial^2 u}{\partial x \partial y} + c \frac{\partial^2 u}{\partial y^2} = f(x,y,u,\partial_x u, \partial_y u)
.\]
Man kann nun Störungen $\tilde{u} = u + u'$ dieser Gleichung betrachten und erhält so eine Fluktuation $u'$ (keine Ableitung) und eine linearisierte Störungsgleichung:
 \[
	 a_0 \frac{\partial^2 u'}{\partial x^2} + 2b_0 \frac{\partial^2 u}{\partial x \partial y} + c_0 \frac{\partial^2 u}{\partial y^2} = 0
.\]
Dabei sind jetzt $a,b,c$ konstant.

Für diese gestörten Gleichungen kann man dann eine Wellenlösung suchen.
Diese existieren unter der Bedingung, dass gilt:
\[
	u' \propto e^{i(n_x x + n_y y)}
.\]
Eingesetzt gelangt man zu einer Gleichung für $n_x$ und $n_y$:
 \[
	an_x^2 + 2b n_x n_y + cn_x^2 = 0
.\]
Und wir können feststellen, dass Wellenlösungen in Abhängigkeit von der Diskriminante $D = b^2 - ac$ existieren:
\begin{itemize}
	\item $D > 0$: zwei unabhängige Wellenlösungen
	\item $D = 0$: degenerierte Welle
	\item $D < 0$: keine Wellenlösung
\end{itemize}

% Minute 24
\section{Systeme erster Ordnung}

\section{Rand- und Anfangsbedingungen}

