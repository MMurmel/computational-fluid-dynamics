\chapter{Einleitung}
\lecture{1}{19.04.2021}{Einleitung}
Ziel dieser Lehrveranstaltung soll sein, die Finite-Differenzen-Methode sowie die Finite-Volumen-Methode kennenzulernen.
Dabei soll das Zusammenspiel von Orts- und Zeitdiskretisierung beleuchtet werden.
Außerdem soll Ziel sein, verschiedene Diskretisierungsmethoden in ihrer Qualität bewerten zu können.

Die numerische Strömungssimulation erfolgt oft in folgender Reihenfolge:
\begin{enumerate}
	\item Problemstellung, physikalische Beschreibung und Vereinfachung 
	\item Mathematisches Modell: DGL mit Anfangs- und Randbedingungen
	\item Numerisches Modell: Orts- und Zeitdiskretisierung
	\item Implementierung
	\item Gittergenerierung
	\item Simulation
	\item Auswertung und Visualisierung
\end{enumerate}
Der dritte Punkt soll den Großteil dieser Vorlesung beanspruchen.
